\documentclass[11pt,a4paper]{article}

% --- Packages ---
\usepackage[utf8]{inputenc} 
\usepackage[T1]{fontenc}    
\usepackage{geometry}       
\geometry{
    a4paper,
    total={170mm,257mm},
    left=25mm,
    top=25mm,
}
\usepackage{graphicx}       
\usepackage{amsmath, amssymb} 
\usepackage{hyperref}       
\hypersetup{                
    colorlinks=true,
    linkcolor=blue,
    urlcolor=cyan,
}
\usepackage{indentfirst}
\usepackage{float}
\usepackage{listings}
\usepackage{xcolor}

\lstset{
    basicstyle=\ttfamily\footnotesize,
    commentstyle=\color{green!60!black},
    keywordstyle=\color{blue},
    numbers=left,
    numberstyle=\tiny\color{gray},
    frame=single,
}

% Title and labeling
\title{Discrete Time Systems \\ [1ex] \large ECSE 313 Group 16 Lab 2 Report}
\author{Nathaniel Hahn and Donovan Crowley}
\date{\today}

% Body
\begin{document}

\maketitle 

\tableofcontents 
\newpage

\section{Results}

% 3.2
\subsection{Background Exercises}

Discrete-time system approximating continuous-time functions differentiator and integrator where T represents the sampling period: \\

Continuous-Time Differentiator:
\[
    y(t) = \frac{d}{dt} x(t)
\]

Discrete-Time Differentiator Equation:
\[
    \frac{dx(t)}{dt} = \frac{x(nT) - x((n-1)T)}{T}
\]

Differentiator Difference Equation:
\[
    y[n] = \frac{x[n] - x[n-1]}{T}
\]

Differentiator Block Diagram:
% ADD DIAGRAM


Continuous-Time Integrator:
\[
    y(t) = \int_{-\infty}^{t} x(\tau) d\tau
\]

Discrete-Time Integrator Equation:
\[
    y[n] = y[n-1] + Tx[n]
\]

Integrator Difference Equation:
\[
    y[n] = y[n-1] + Tx[n]
\]

Integrator Block Diagram:
% ADD DIAGRAM

\textbf{Equation 3.3}
\[
    y[n] = \frac{x[n] + x[n-1] + x[n-2]}{3}
\]

% ADD DIAGRAM
\[
    h[0] = \frac{1 + 0 + 0}{3} = \frac{1}{3}
\]\[
    h[1] = \frac{0 + 1 + 0}{3} = \frac{1}{3}
\]\[
    h[2] = \frac{0 + 0 + 1}{3} = \frac{1}{3}
\]\[
    h[n] = \frac{\delta[n] + \delta[n-1] + \delta[n-2]}{3}
\]

\textbf{Equation 3.4}
\[
    y[n] = 0.8y[n-1] + 0.2x[n]
\]

% ADD DIAGRAM

\[
    h[0] = 0.8(0) + 0.2(1) = 0.2
\]\[
    h[1] = 0.8(0.2) + 0.2(0) = 0.2 * 0.8
\]\[
    h[2] = 0.8(0.2 * 0.8) + 0.2(0) = 0.2*(0.8)^2
\]\[
    h[n] = 0.2(0.8)^nu[n]
\]

\textbf{Equation 3.5}
\[
    y[n] = y[n-1] + \frac{x[n] - x[n-3]}{3}
\]

% ADD DIAGRAM

\[
    h[0] = 0 + \frac{1 - 0}{3} = \frac{1}{3}
\]\[
    h[1] = \frac{1}{3} + \frac{0}{3} = \frac{1}{3}
\]\[
    h[2] = \frac{1}{3} + \frac{0}{3} = \frac{1}{3}
\]\[
    h[3] = \frac{1}{3} + \frac{0 - 1}{3} = 0
\]
\[
    h[4] = 0 + \frac{0}{3} = 0
\]
\[
    h[n] = \frac{\delta[n] + \delta[n-1] + \delta[n-2]}{3}
\]

A "moving average" is described as a sliding window of a specific length the efficiently smooths signals by removing noise while maintaining the original properties.
Equations 3.3 and 3.5 are known as moving averages because they both are a 3-point windows that calculate the mean of the three most recent samples. 

% 3.3
\subsection{Example Discrete-Time Systems}
\graphicspath{{3.3/}}
\begin{figure}[H]
    \centering
    \includegraphics[width=0.9\textwidth]{image1.png}
    \caption{Discrete-Time Systems applying the differentiator and integrator systems}
\end{figure}
In Figure 1, the integrator system is unstable because there exists at least one bounded input that results in an unbounded input as seen in subplot 5.
As n increases in subplot 5, the output stays at a constant value, and therefore the system is unstable.
However, the differentiator system is BIBO stable because the finite input values will always be bounded and the output will never grow to infinity.
Even for in subplot 6, the unit step value is derived to a bounded dirac delta function.

% 3.4
\subsection{Difference Equations}
\graphicspath{{3.4/}}
\begin{figure}[H]
    \centering
    \includegraphics[width=0.9\textwidth]{plots_3_4.jpg}
    \caption{}
\end{figure}
% ADD SOME OBSERVATIONS HERE and system diagrams!!!

% 3.5
\subsection{Audio Filtering}

% 3.6
\subsection{Inverse Systems}

% 3.7
\subsection{System Tests}

% 3.8
\subsection{Stock Market Example}
\graphicspath{{3.8/}}
\begin{figure}[H]
    \centering
    \includegraphics[width=0.9\textwidth]{image3_8.png}
    \caption{}
\end{figure}

\end{document}
