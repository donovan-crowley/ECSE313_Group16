\documentclass[11pt,a4paper]{article}

% --- Packages ---
\usepackage[utf8]{inputenc} 
\usepackage[T1]{fontenc}    
\usepackage{geometry}       
\geometry{
    a4paper,
    total={170mm,257mm},
    left=25mm,
    top=25mm,
}
\usepackage{graphicx}       
\usepackage{amsmath, amssymb} 
\usepackage{hyperref}       
\hypersetup{                
    colorlinks=true,
    linkcolor=blue,
    urlcolor=cyan,
}
\usepackage{indentfirst}
\usepackage{float}
\usepackage{listings}
\usepackage{xcolor}

\lstset{
    basicstyle=\ttfamily\footnotesize,
    commentstyle=\color{green!60!black},
    keywordstyle=\color{blue},
    numbers=left,
    numberstyle=\tiny\color{gray},
    frame=single,
}

% Title and labeling
\title{Discrete Time Systems \\ [1ex] \large ECSE 313 Group 16 Lab 2 Report}
\author{Nathaniel Hahn and Donovan Crowley}
\date{\today}

% Body
\begin{document}

\maketitle 

\tableofcontents 
\newpage

\section{Results}

% 3.2
\subsection{Background Exercises}

Discrete-time system approximating continuous-time functions differentiator and integrator where T represents the sampling period: \\

Continuous-Time Differentiator:
\[
    y(t) = \frac{d}{dt} x(t)
\]

Discrete-Time Differentiator Equation:
\[
    \frac{dx(t)}{dt} = \frac{x(nT) - x((n-1)T)}{T}
\]

Differentiator Difference Equation:
\[
    y[n] = \frac{x[n] - x[n-1]}{T}
\]

Continuous-Time Integrator:
\[
    y(t) = \int_{-\infty}^{t} x(\tau) d\tau
\]

Discrete-Time Integrator Equation:
\[
    y[n] = y[n-1] + Tx[n]
\]

Integrator Difference Equation:
\[
    y[n] = y[n-1] + Tx[n]
\]

\graphicspath{{3.2/}}
\begin{figure}[H]
    \centering
    \includegraphics[width=0.9\textwidth]{figure_323.png}
    \caption{System Diagrams Representing Continuous and Discrete-Time Integrators and Differentiators}
\end{figure}

\textbf{Equation 3.3}
\[
    y[n] = \frac{x[n] + x[n-1] + x[n-2]}{3}
\]

\[
    h[0] = \frac{1 + 0 + 0}{3} = \frac{1}{3}
\]\[
    h[1] = \frac{0 + 1 + 0}{3} = \frac{1}{3}
\]\[
    h[2] = \frac{0 + 0 + 1}{3} = \frac{1}{3}
\]\[
    h[n] = \frac{\delta[n] + \delta[n-1] + \delta[n-2]}{3}
\]

\textbf{Equation 3.4}
\[
    y[n] = 0.8y[n-1] + 0.2x[n]
\]

\[
    h[0] = 0.8(0) + 0.2(1) = 0.2
\]\[
    h[1] = 0.8(0.2) + 0.2(0) = 0.2 * 0.8
\]\[
    h[2] = 0.8(0.2 * 0.8) + 0.2(0) = 0.2*(0.8)^2
\]\[
    h[n] = 0.2(0.8)^nu[n]
\]

\textbf{Equation 3.5}
\[
    y[n] = y[n-1] + \frac{x[n] - x[n-3]}{3}
\]

\[
    h[0] = 0 + \frac{1 - 0}{3} = \frac{1}{3}
\]\[
    h[1] = \frac{1}{3} + \frac{0}{3} = \frac{1}{3}
\]\[
    h[2] = \frac{1}{3} + \frac{0}{3} = \frac{1}{3}
\]\[
    h[3] = \frac{1}{3} + \frac{0 - 1}{3} = 0
\]
\[
    h[4] = 0 + \frac{0}{3} = 0
\]
\[
    h[n] = \frac{\delta[n] + \delta[n-1] + \delta[n-2]}{3}
\]

\begin{figure}[H]
    \centering
    \includegraphics[width=0.9\textwidth]{equations_diagrams.png}
    \caption{System Diagrams For Equations 3.3, 3.4, and 3.5}
\end{figure}


A "moving average" is described as a sliding window of a specific length the efficiently smooths signals by removing noise while maintaining the original properties.
Equations 3.3 and 3.5 are known as moving averages because they both are a 3-point windows that calculate the mean of the three most recent samples. 

% 3.3
\subsection{Example Discrete-Time Systems}
\graphicspath{{3.3/}}
\begin{figure}[H]
    \centering
    \includegraphics[width=0.9\textwidth]{image1.png}
    \caption{Discrete-Time Systems Applying the Differentiator and Integrator Systems}
\end{figure}
In Figure 3, the integrator system is unstable because there exists at least one bounded input that results in an unbounded input as seen in subplot 5.
As n increases in subplot 5, the output stays at a constant value, and therefore the system is unstable.
However, the differentiator system is BIBO stable because the finite input values will always be bounded and the output will never grow to infinity.
Even for in subplot 6, the unit step value is derived to a bounded dirac delta function.

% 3.4
\subsection{Difference Equations}
\graphicspath{{3.4/}}
\begin{figure}[H]
    \centering
    \includegraphics[width=0.9\textwidth]{plots_3_4.jpg}
    \caption{S1 and S2 Impulse Responses and Properties}
\end{figure}
In Figure 4, subplots 3 and 4 are identical because S1 and S2 are Linear Time-Invariant systems that when convolved with each other are equivalent under the commutative property.
Additionally, subplot 5 follows the additive property where the result is a direct addition between the dirac delta functions in subplot 1 and subplot 2.

\begin{figure}[H]
    \centering
    \includegraphics[width=0.9\textwidth]{input.png}
    \caption{S1 and S2 Impulse System Diagrams}
\end{figure}

\begin{figure}[H]
    \centering
    \includegraphics[width=0.9\textwidth]{image2.png}
    \caption{S1(S2) Convolution System Diagram}
\end{figure}

\begin{figure}[H]
    \centering
    \includegraphics[width=0.9\textwidth]{image3.png}
    \caption{S2(S1) Convolution System Diagram}
\end{figure}

\begin{figure}[H]
    \centering
    \includegraphics[width=0.9\textwidth]{image4.png}
    \caption{S1 + S2 System Diagram}
\end{figure}


% 3.5
\subsection{Audio Filtering}
The S1 filter from Section 1.3 produces a higher pitch and the S2 filter has a lower pitch.
The reason for this is because the two filters represent high-pass and low-pass filters. \\

Since S1 represents a high-pass filter where the function calculates the difference between adjacent samples, the audio has a higher pitch as waves moving at higher frequences have large differences among them.
Thus higher frequencies pass through. \\

Conversely, S2 represents a low-pass filter that recursively smooths the average current input with previous input, restricting higher frequencies to pass through.
Moreover, lower frequency signal's properties are preserved and are played in the audio.

% 3.6
\subsection{Inverse Systems}
\graphicspath{{3.6/}}
\begin{figure}[H]
    \centering
    \includegraphics[width=0.9\textwidth]{plots.jpg}
    \caption{Impulse Responses of S2, S3, and S3(S2)}
\end{figure}

\begin{figure}[H]
    \centering
    \includegraphics[width=0.9\textwidth]{image1.png}
    \caption{System Diagram of S3}
\end{figure}


% 3.7
\subsection{System Tests}
\graphicspath{{3.7/}}
\begin{figure}[H]
    \centering
    \includegraphics[width=0.9\textwidth]{plots.jpg}
    \caption{Input/Output Signals For System1, System2, and System3}
\end{figure}
In Figure 11, the first subplot plotting system1 shows a time-variant system where the two functions plotted look similar in shape, but the delayed input of S reults in a simple shift of the original output, indicating the function is depdendent on time.
Additionally, the third subplot with system3 plots a non-linear function where the two curves present different shapes and amplitudes, indicating the output did not scale properly after the addition of the two signals.

% 3.8
\subsection{Stock Market Example}
\graphicspath{{3.8/}}
\begin{figure}[H]
    \centering
    \includegraphics[width=0.9\textwidth]{image3_8.png}
    \caption{Original Stock Market and Filtered Exchange-Rates}
\end{figure}
In Figure 12, subplot 2 using Equation 3.4 presents a smoother curve that is ideal for finding long-term trends in the stock market.
However, subplot 2 is not accurate for the sporadic, short-term behavior of the stock rates.
Subplot 3 is a lighter filter that adjusts to the real-time price while maintaining a discernible pattern.

\end{document}
